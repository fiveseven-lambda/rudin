\documentclass[uplatex]{jsarticle}
\usepackage{amsmath, amssymb}
\begin{document}
\section{1.11}
	Suppose $S$ is an ordered set with the least-upper-bound property,
	$B \subset S$,
	$B$ is not empty,
	and $B$ is bounded below.

	Let $L$ be the set of all lower bounds of $B$. $L \subset S$.

	Because $B$ is bounded below, $L$ is not empty.

	Take $b \in B$.
	For all $l \in L$, $l$ is a lower bound of $B$,
	so $l \leq b$ holds.
	This means $b$ is an upper bound of $L$.
	Because $B$ is not empty,
	$b \in B$ exists
	and $L$ is bounded above.

	Due to the least-upper-bound property of $S$,
	$\sup L$ exists:
	$\sup L$ is an upper bound of $L$,
	and any $\gamma < \sup L$ is not an upper bound of $L$.

	For any $b \in B$,
	$b$ is an upper bound of $L$,
	so $\sup L \leq b$.
	This means $\sup L$ is a lower bound of $B$,
	and $\sup L \in L$.

	Because $\sup L$ is an upper bound of $L$,
	any $l \in L$ satisfies $l \leq \sup L$.

	So, $\sup L$ is a lower bound of $B$,
	and any $l > \sup L$ is not a lower bound of $B$.
	This means $\sup L = \inf B$.
\section{2.23}
\newcommand{\co}{\mathrm{c}}
	By definition,
	$E^\co$ is closed if and only if
	any limit point of $E^\co$ is not a point of $E$.
	This is equivalent to its contrapositive:
	any $p \in E$ is not a limit point of $E^\co$.
	By the definition of a limit point,
	this means that every $p \in E$
	has at least one neighborhood $N_r(p)$ such that
	every $q \in N_r(p)$ satisfies $q \in E$
	as long as $p \neq q$.
	The constraint ``as long as $p \neq q$" is
	actually unnecessary,
	because $p$ itself also satisfies $p \in E$.
	So, this can be restated as
	every $p \in E$ has at least one neighborhood $N_r(p)$
	such that $N_r(p) \subset E$.
	This is exactly the definition of an open set.
\section{2.34}
	Let $(X, d)$ be a metric space
	and $K$ a compact subset of $X$.
	Take $p \in K^\co$.
	For each $r > 0$,
	define $V_r = \{q \in X \mid d(p, q) > r \}$,
	then $V_r \subset V_{r'}$ if $r > r'$.
	Since $\bigcup_r V_r = X \setminus \{p\} \supset K$ and $K$ is compact,
	we can choose $r_1, \ldots, r_n > 0$
	with $K \subset \bigcup_{i = 1}^n V_{r_i}$.
	Let $r_{\rm min} = \min\{r_1, \ldots, r_n\}$,
	then $K \subset V_{r_{\rm min}}$,
	hence $N_{r_{\rm min}}(p) \subset K^\co$.
	Therefore $K^\co$ is open
	and $K$ is closed.
\end{document}
