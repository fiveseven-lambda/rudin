\documentclass[uplatex]{jsarticle}
\begin{document}
\section{1.11}
	Suppose $S$ is an ordered set with the least-upper-bound property,
	$B \subset S$,
	$B$ is not empty,
	and $B$ is bounded below.

	Let $L$ be the set of all lower bounds of $B$. $L \subset S$.

	Because $B$ is bounded below, $L$ is not empty.

	Take $b \in B$.
	For all $l \in L$, $l$ is a lower bound of $B$,
	so $l \leq b$ holds.
	This means $b$ is an upper bound of $L$.
	Because $B$ is not empty,
	$b \in B$ exists
	and $L$ is bounded above.

	Due to the least-upper-bound property of $S$,
	$\sup L$ exists:
	$\sup L$ is an upper bound of $L$,
	and any $\gamma < \sup L$ is not an upper bound of $L$.

	For any $b \in B$,
	$b$ is an upper bound of $L$,
	so $\sup L \leq b$.
	This means $\sup L$ is a lower bound of $B$,
	and $\sup L \in L$.

	Because $\sup L$ is an upper bound of $L$,
	any $l \in L$ satisfies $l \leq \sup L$.

	So, $\sup L$ is a lower bound of $B$,
	and any $l > \sup L$ is not a lower bound of $B$.
	This means $\sup L = \inf B$.
\end{document}
